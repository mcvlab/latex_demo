% 来源: https://zhuanlan.zhihu.com/p/530865615


\documentclass{ctexart}
\CTEXsetup[format={\Large\bfseries}]{section}%让section指令左对齐
\renewcommand{\abstractname}{}%去掉摘要头上的标题
\newcommand{\upcite}[1]{\textsuperscript{\textsuperscript{\cite{#1}}}}%右上角引用文献的命令
\usepackage[margin=2cm]{geometry}%调整页边距
\usepackage{pifont}%提供圆圈数字输入
\usepackage{graphicx}%插入图片
\usepackage{authblk}%作者、单位
\usepackage{amsmath, bm}%数学公式宏包
\usepackage{esint}%使重积分符号更加紧凑,必须加在amsmath后
\usepackage{amssymb}%特殊数学符号
\usepackage{caption}%图片标题处理
\usepackage{float}%处理图表浮动插入
\usepackage[section]{placeins}%防止图表浮动跨过section
\usepackage{subfigure}%插入多图时用子图显示的宏包
\pagestyle{plain}%页码
\setCJKfamilyfont{zhsong}[AutoFakeBold = {2.17}]{SimSun}
% \setCJKmainfont{SimSun}[BoldFont=FandolSong-Bold]
\renewcommand*{\songti}{\CJKfamily{zhsong}}%定义新宋体命令
\setlength{\belowcaptionskip}{-2pt}
\begin{document}

	\zihao{5}%设置全文字号为五号
	\everymath{\displaystyle}%设置所有数学公式为displaystyle形式
	\abovedisplayshortskip=5pt%设置数学公式间距
	\belowdisplayshortskip=5pt
	\abovedisplayskip=5pt
	\belowdisplayskip=5pt
	\lineskiplimit=4pt
	\lineskip=4pt
	\title{\vspace{-2cm}{\heiti {\zihao{2}论文标题}} }%标题
	\date{}%不显示日期
	\author[1]{\zihao{-4}作者1 , 作者2 , 作者3\vspace{-1em}}%作者名称
	\affil[1]{\vspace{-3em}{{\zihao{6}{\kaishu (单位)}}}}%作者单位

	\maketitle 
	\begin{abstract}
		% \newgeometry{left=1.5cm, right=1.5cm}%调整摘要部分的页边距,与正文对齐
		\noindent{\zihao{-5}{\heiti 摘~~~要 }{\kaishu abstract is here。}}\\
		\noindent{\zihao{-5}\heiti 关键词 }~~~{\zihao{-5}\kaishu 关键词1~~~~关键词2~~~~关键词3}\\
		\\
	\end{abstract}

	\vspace{-2em}
	\section{引言}

	introduction is here.

	paragraph paragraph paragraph paragraph paragraph paragraph paragraph paragraph paragraph paragraph paragraph paragraph paragraph paragraph paragraph paragraph

	paragraph

	paragraph

	paragraph

	paragraph

	paragraph

	\section{相关工作}

	related work is here.

	paragraph

	paragraph

	paragraph

	paragraph

	paragraph

	paragraph

	\section{方法}

	method is here.

	paragraph

	paragraph

	paragraph

	paragraph

	paragraph

	paragraph

	\section{实验}

	experiments are here.

	paragraph

	paragraph

	paragraph

	paragraph

	paragraph

	paragraph

	\section{结论}

	conclusions are here.

	paragraph

	paragraph

	paragraph

	paragraph

	paragraph

	paragraph
	
	%如果你会用BibTeX,请使用生成参考文献列表的命令
	%\bibliographystyle{unsrt}
	%\bibliography{ref} 
	
\end{document}	